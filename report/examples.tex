% File: examples.tex
% Date: Fri Jun 01 16:03:27 2012 +0800
% Author: Yuxin Wu <ppwwyyxxc@gmail.com>
\section{examples}
\label{sec:exam}



	\lstset{numbers=left, numberstyle=\tiny}
\begin{lstlisting}[mathescape]
for every pixel $ (i,j)$ {
  z_max = $ -\infty $
  for every polygon $ P_k${
    if ( $ (i.j) \in P_k $ ){
      depth = depth of $ P_k$ at $ (i,j)$
      if (depth > z_max)
        z_max = depth, index_p = k;
    }
  }
  if (z_max == $ -\infty$) draw(i, j, BackgroundColor);
  else draw(i, j, ColorOf $ P_k$ at $ (i,j)$)
}
\end{lstlisting}

\hyperref[sec:exam]{掃描線}\cite[p.233]{example}


	\begin{figure*}[!t]
		\centering
		\subfigure[信息缺失] {\includegraphics[scale=0.4]{res/a.png}}
		\subfigure[查询文化素质课] {\includegraphics[scale=0.5]{res/a.png}}
		\caption{清华大学选课系统用户体验的几个常见弊端\label{fig:shortage}}
	\end{figure*}


	
%\begin{figure}[h]
	%\centering
	%\begin{tikzpicture}
		%\tikzset{edge from parent/.style={}}
		%\Tree [.\node(web){website}; [.\node(srv){server};
		%\node(j0){judge$_0$}; \node(j1){judge$_1$}; \node(jdot){$\ldots$}; \node(jn){judge$_{n-1}$}; ] ]
		%\draw[->] (srv) to (web);
		%\draw[->] (j0) to [out=80,in=-160] (srv);
		%\draw[->] (j1) to [out=90,in=-120] (srv);
		%\draw[->] (jdot) to [out=100,in=-80] (srv);
		%\draw[->] (jn) to [out=110,in=-40] (srv);
	%\end{tikzpicture}
	%\caption{总体构架\label{fig:overall-structure}}
%\end{figure}



\newcommand{\judgeitem}[3]{
	[{\bf #1}]得分 & #2 \\\hline
<{\bf 扣分依据}> & #3 \\\hline
}

\begin{table}[h]
	\centering
	\begin{tabular}{|l|l|}
		\hline
		\judgeitem{基本要求}{10}{运行良好}
		\judgeitem{代码格式}{10}{}
		\judgeitem{符号名称}{10}{命名规范易懂}
		\judgeitem{算法正确}{10}{}
		\judgeitem{设计合理}{10}{结构清晰,层次分明}
	\end{tabular}
\end{table}
