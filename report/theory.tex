% File: theory.tex
% Date: Sun Aug 26 20:59:52 2012 +0800
% Author: Yuxin Wu <ppwwyyxxc@gmail.com>
\section{Algorithms}
\label{sec:algorithms}

{\bf 定义:}复平面上使得数列$ \{z_n\}: z_0 = 0; z_{n+1} = z_n^2 + c$收敛的全体复数c的集合称为Mandelbrot Set.

{\bf 定理:}$ \{z_n\}: z_0 = 0; z_{n+1} = z_n^2 + c $ 中有$ z_k $满足$ |z_k| > 2$,则$ \{ z_n\}$必发散.

{\bf 证明:}

首先可以看出,必存在$ z_k$满足$ |z_k| \ge |c|$且$ |z_k| > 2$. 当$ |c| \le 2$时显然存在,因为存在$ |z_k| > 2$.
当$ |c| > 2$时,$ |z_1| = |c| \ge |c|$满足条件.

于是,选取一个$ z_k$,满足$ |z_k| \ge |c|, |z_k| > 2$,那么一定$ \exists t > 0, |z_k| > 2 + t.$

我们可以得到如下的不等式:
\[ |z_{k+1}| = |z_k^2 + c| \ge |z_k^2| - |c| \ge |z_k^2| - |z_k| = (|z_k| - 1)|z_k| > (1 + t)|z_k| \]

迭代下去,有
\[ |z_{k+2}| = |z_{k+1}^2 + c| \ge |z_{k+1}^2| - |c| \ge |z_{k+1}^2| - |z_{k+1}| > (1+t)|z_{k+1}| > (1+t)^2|z_k|\]

$ \cdots \cdots$ 

于是,$ |z_{k+n}| > (1+t)^n|z_k|.$由于$ 1 + t > 1, $显然数列发散. 得证.
\\

由上述定理,计算一点$ c = c_r + c_ii$是否在Mandelbrot Set中,在迭代时一旦发现超出复平面上半径为2的圆,就可直接退出循环,大幅减少迭代时间.算法代码如下:
\txtsrcpart{res/src/calculate.cc}{40}{51}

利用类似手段,还可以得到一个结论:若$ |c| < \dfrac{1}{4}, $则数列必发散.
经实验,这个结论由于只对小范围数据有效,因此添加到程序中后(注释掉的代码)大多数情形下会影响程序效率.
